\documentclass[10pt]{article}

\usepackage{mathtools, amssymb, bm}
\usepackage{microtype}
\usepackage[utf8]{inputenc}
\usepackage[margin = 0.75in]{geometry}
\usepackage{booktabs}
\usepackage{graphicx}
\usepackage{xcolor}
\usepackage{tikzsymbols}
\usepackage[hidelinks]{hyperref}
\usepackage{titlesec}



% \titleformat{\section}{\normalsize\bfseries}{\thesection}{1em}{}
\titleformat{\section}{\large\bfseries}{\thesection}{1em}{}
\setcounter{secnumdepth}{0}

\definecolor{colabcol}{HTML}{960018}
\newcommand{\mycolab}[1]{\textcolor{colabcol}{\textsl{Collaborators:}} #1 \\ }
\newcommand{\mycolaba}[1]{\textcolor{colabcol}{\textsl{Collaborators:}} #1}

\title{
    {\Large Homework 2}
}
\author{
    {\normalsize Aiden Kenny}\\
    {\normalsize STAT GR5204: Statistical Inference}\\
    {\normalsize Columbia University}
}
\date{\normalsize Novermber 26, 2020}

\begin{document}

\maketitle

%' ============================================================================================================================================================
\section{Question 1} \noindent


%' ============================================================================================================================================================
\section{Question 9} \noindent
If \(X_1, \ldots, X_n \overset{\mathrm{iid}}{\sim} \mathrm{N}(\mu, \sigma^2)\) with unknown \(\mu\) and \(\sigma^2\), then a \(\gamma\)\%{} confidence interval
for \(\mu\) is given by 
\begin{align*}
    \mathcal{I} = \Big( ~\bar{X} - t_{\gamma}(n) \cdot S / \sqrt{n} ~,~ \bar{X} + t_{\gamma}(n) \cdot S / \sqrt{n} ~ \Big),
\end{align*}
where \(t_{\gamma}(n) = T_{n-1}^{-1} ( (1 + \gamma) / 2 )\) is the \((1 + \gamma) / 2\)th quantile of the \(t\) distribution with \(\mathrm{df} = n-1\) and \(S\)
is the sample standard deviation. 
The length of this confidence interval is given by 
\begin{align*}
    \Delta 
    % = \bar{X} + T_{n-1}^{-1}\big( (1 + \gamma) / 2 \big) \frac{S}{\sqrt{n}} - \bigg( \bar{X} - T_{n-1}^{-1}\big( (1 + \gamma) / 2 \big) \frac{S}{\sqrt{n}} \bigg)
    = \max (\mathcal{I}) - \min (\mathcal{I})
    = \Big( \bar{X} + t_{\gamma}(n) \cdot S / \sqrt{n} \Big) - \Big( \bar{X} - t_{\gamma}(n) \cdot S / \sqrt{n} \Big)
    = 2 t_{\gamma}(n) \cdot S / \sqrt{n}.
\end{align*}
The squared length is then given by \(\Delta^2 = 4 t_{\gamma}^2(n) \cdot S^2 / n\). 
Because the sample variance is an unbiased estimator for \(\sigma^2\), we have \(\mathbb{E}[\Delta^2] = \mathbb{E} \big[ 4 t_{\gamma}^2(n) \cdot S^2 / n \big] = 4 t_{\gamma}^2(n) \cdot \sigma^2 / n\).
We now set \(\mathbb{E}[\Delta^2] < \sigma^2 / 2\), and after some cancellations, we see that we need \(t_{\gamma}^2(n) / n < 1/8\). There is no way to find a
closed-form expression for this, so we will have to check the value of \(t_{\gamma}^2(n) / n\) for increasing values of \(n\). I set up a \texttt{while} 
loop in \texttt{R} to solve for it, and when \(\gamma = 0.9\), we find that \(n = 24\) is the smallest value of \(n\) such that \(\mathbb{E}[\Delta^2] < \sigma^2 / 2\).



\end{document}