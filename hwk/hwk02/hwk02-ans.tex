\documentclass[10pt]{article}

\usepackage{mathtools, amssymb, bm}
\usepackage{microtype}
\usepackage[utf8]{inputenc}
\usepackage[margin = 0.75in]{geometry}
\usepackage{booktabs}
\usepackage{graphicx}
\usepackage{xcolor}
\usepackage{tikzsymbols}
\usepackage[hidelinks]{hyperref}
\usepackage{titlesec}



% \titleformat{\section}{\normalsize\bfseries}{\thesection}{1em}{}
\titleformat{\section}{\large\bfseries}{\thesection}{1em}{}
\setcounter{secnumdepth}{0}

\definecolor{colabcol}{HTML}{960018}
\newcommand{\mycolab}[1]{\textcolor{colabcol}{\textsl{Collaborators:}} #1 \\ }
\newcommand{\mycolaba}[1]{\textcolor{colabcol}{\textsl{Collaborators:}} #1}

\title{
    {\Large Homework 2}
}
\author{
    {\normalsize Aiden Kenny}\\
    {\normalsize STAT GR5204: Statistical Inference}\\
    {\normalsize Columbia University}
}
\date{\normalsize Novermber 26, 2020}

\begin{document}

\maketitle

%' ============================================================================================================================================================
\section{Question 1} \noindent


%' ============================================================================================================================================================
\section{Question 9} \noindent
If \(X_1, \ldots, X_n \overset{\mathrm{iid}}{\sim} \mathrm{N}(\mu, \sigma^2)\) with unknown \(\mu\) and \(\sigma^2\), then a \(\gamma\)\%{} confidence interval
for \(\mu\) is given by 
\begin{align*}
    \mathcal{I} = \Big( ~\bar{X} - t_{\gamma}(n) \cdot S / \sqrt{n} ~,~ \bar{X} + t_{\gamma}(n) \cdot S / \sqrt{n} ~ \Big),
\end{align*}
where \(t_{\gamma}(n) = T_{n-1}^{-1} ( (1 + \gamma) / 2 )\) is the \((1 + \gamma) / 2\)th quantile of the \(t\) distribution with \(\mathrm{df} = n-1\) and \(S\)
is the sample standard deviation. 
The length of this confidence interval is given by 
\begin{align*}
    \Delta 
    % = \bar{X} + T_{n-1}^{-1}\big( (1 + \gamma) / 2 \big) \frac{S}{\sqrt{n}} - \bigg( \bar{X} - T_{n-1}^{-1}\big( (1 + \gamma) / 2 \big) \frac{S}{\sqrt{n}} \bigg)
    = \max (\mathcal{I}) - \min (\mathcal{I})
    = \Big( \bar{X} + t_{\gamma}(n) \cdot S / \sqrt{n} \Big) - \Big( \bar{X} - t_{\gamma}(n) \cdot S / \sqrt{n} \Big)
    = 2 t_{\gamma}(n) \cdot S / \sqrt{n}.
\end{align*}
The squared length is then given by \(\Delta^2 = 4 t_{\gamma}^2(n) \cdot S^2 / n\). 
Because the sample variance is an unbiased estimator for \(\sigma^2\), we have \(\mathbb{E}[\Delta^2] = \mathbb{E} \big[ 4 t_{\gamma}^2(n) \cdot S^2 / n \big] = 4 t_{\gamma}^2(n) \cdot \sigma^2 / n\).
We now set \(\mathbb{E}[\Delta^2] < \sigma^2 / 2\), and after some cancellations, we see that we need \(t_{\gamma}^2(n) / n < 1/8\). There is no way to find a
closed-form expression for this, so we will have to check the value of \(t_{\gamma}^2(n) / n\) for increasing values of \(n\). I set up a \texttt{while} 
loop in \texttt{R} to solve for it, and when \(\gamma = 0.9\), we find that \(n = 24\) is the smallest value of \(n\) such that \(\mathbb{E}[\Delta^2] < \sigma^2 / 2\).

%' ============================================================================================================================================================
\section{Question 10} \noindent
Let \(\bm{X} \overset{\mathrm{iid}}{\sim} \mathrm{N}(\theta, \sigma^2)\), where \(\theta\) is unknown and \(\sigma^2\) is known, and we assume prior that 
\(\theta \sim \mathrm{N}(\mu, \nu^2\), where both \(\mu\) and \(\nu^2\) are known. 
\begin{itemize}
    \item[(a)] Since normal distributions are are conjugate to normal sampling, it follows
    that \(\theta \,|\, \mathbf{x} \sim \mathrm{N}(\tilde{\mu}, \tilde{\sigma}^2)\), where 
    \begin{align*}
        \tilde{\mu}
        = \frac{\sigma^2 \mu + n \nu^2 \bar{x}}{\sigma^2 n \nu^2}
        ~~~\text{and}~~~
        \tilde{\sigma}^2
        = \frac{\sigma^2 \nu^2}{\sigma^2 + n \mu^2}.
    \end{align*}
    We also know that \((\theta \,|\, \mathbf{x} - \tilde{\mu}) / \tilde{\sigma} \sim \mathrm{N}(0,1)\), and so a \(95\)\%{} confidence interval for 
    \(\theta \,|\, \mathbf{x}\) is given by 
    \begin{align*}
        \mathcal{I}
        = \Big( ~ \tilde{\mu} - \Phi^{-1}(0.975) \cdot \tilde{\sigma} ~,~ \tilde{\mu} + \Phi^{-1}(0.975) \cdot \tilde{\sigma} ~ \Big).
        % = \Big( ~ \frac{\sigma^2 \mu + n \nu^2 \bar{x}}{\sigma^2 n \nu^2} - \Phi^{-1}(0.975) \cdot \frac{\sigma^2 \nu^2}{\sigma^2 + n \mu^2} ~,~ \frac{\sigma^2 \mu + n \nu^2 \bar{x}}{\sigma^2 n \nu^2} + \Phi^{-1}(0.975) \cdot \frac{\sigma^2 \nu^2}{\sigma^2 + n \mu^2} ~ \Big)
    \end{align*} 
    \item[(b)] We can think of our interval \(\mathcal{I}\) as a function of \(\nu^2\). To examine what happens to \(\mathcal{I}(\mu^2)\) as \(\nu^2 \to \infty\), we 
    will first look at \(\tilde{\mu}\) and \(\tilde{\sigma}\). Using L'Hopital's rule, we have
    \begin{align*}
        \lim_{\nu^2 \to \infty} \tilde{\mu}
        &= \lim_{\nu^2 \to \infty} \frac{\sigma^2 \mu + n \nu^2 \bar{x}}{\sigma^2 n \nu^2}
        = \lim_{\nu^2 \to \infty} \frac{n \bar{x}}{n}
        = \bar{x}, \\
        \lim_{\nu^2 \to \infty} \tilde{\sigma}
        &= \lim_{\nu^2 \to \infty} \sqrt{\frac{\sigma^2 \nu^2}{\sigma^2 + n \mu^2}}
        = \sqrt{ \lim_{\nu^2 \to \infty} \frac{\sigma^2 \nu^2}{\sigma^2 + n \mu^2}}
        = \sqrt{ \lim_{\nu^2 \to \infty} \frac{\sigma^2}{n}}
        % = \sqrt{\frac{\sigma^2}{n}}
        = \frac{\sigma}{\sqrt{n}},
    \end{align*}
    and so \(\mathcal{I}(\nu^2) \to \big( ~\bar{x} - \Phi^{-1}(0.975) \cdot \sigma / \sqrt{n} ~,~ \bar{x} + \Phi^{-1}(0.975) \cdot \sigma / \sqrt{n} ~ \big)\),
    which is a \(95\)\%{} confidence interval for \(\theta\). 
\end{itemize}



\end{document}